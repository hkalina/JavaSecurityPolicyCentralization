\chapter{Obsah CD}

\begin{itemize}
  \item {\tt projekt.pdf} -- Technická zpráva ve formátu PDF
  \item {\tt projekt} -- Zdrojové soubory technické zprávy ve formátu \LaTeX

  \item {\tt wildfly-8.0.0.Final} -- Instalace WildFly s nainstalovanými implementovanými rošířeními, použitelná pro předvádění bakalářské práce

  \item {\tt jsm-policy-subsystem} -- Subsystém WildFly implementovaný jako hlavní součást této práce
  \item {\tt jsm-policy-console-hal} -- Rozšíření webové konzole WildFly umožňující správu výše uvedeného rozšíření (popsané v kapitole \ref{navrhGUI})
  \item {\tt jsm-policy-test} -- Repozitář integračních testů popsaných v kapitole \ref{testovani}

  \item {\tt jboss-modules} -- Záplata popsaná v kapitole \ref{upravaZavadeceWildFly} v odpovídajícím Git repozitáři
\end{itemize}


\chapter{Postup instalace}

Uvedený postup předpokládá operační systém Linux s nainstalovaným běhovým prostředím Javy. Testován byl na distribuci Ubuntu 13.10 s běhovým prostředím Javy OpenJDK 7.

\begin{enumerate}
  \item Připravte standardní instalaci aplikačního serveru WildFly. Doporučenou verzí je 8.0.0.Final. Postup by měl fungovat i pro novější verze.
  
  \item Proveďte instalaci záplaty {\tt jboss-modules}: (Její účel je popsán v kapitole \ref{zmenaZaBehu})
  \begin{enumerate}
    \item Nakopírujte do zapisovatelného adresáře repozitář {\tt jboss-modules}. Získat jej můžete ze stejně pojmenovaného adresáře na přiloženém CD nebo z GitHub:
      \newline\url{https://github.com/honza889/jboss-modules}
    \item Proveďte kompilaci za pomoci nástroje Maven -- v adresáři {\tt jboss-modules} spusťte: {\tt mvn install}
    \item Výsledným archivem ({\tt target/jboss-modules-*-SNAPSHOT.jar}) nahraďte archiv {\tt jboss-modules.jar} v adresáři WildFly.
    \item Na začátek spouštěcího skriptu, kterým budete WildFly spouštět, (např. {\tt bin/dom ain.sh}) přidejte:
      \begin{lstlisting}
JAVA_OPTS="$JAVA_OPTS -Djboss.modules.policy-refreshable=true"
      \end{lstlisting}
    \item Po dalším startu by již WildFly měl používat dynamická oprávnění. (viz kapitola \ref{staticPerm})
  \end{enumerate}
  
  \item Proveďte instalaci rozšíření WildFly {\tt jsm-policy-subsystem}:
  \begin{enumerate}
    \item Nakopírujte do zapisovatelného adresáře repozitář {\tt jsm-policy-subsystem}. Získat jej můžete opět z přiloženého CD nebo z GitHub:
      \newline\url{https://github.com/honza889/jsm-policy-subsystem}
    \item Proveďte kompilaci za pomoci nástroje Maven: {\tt mvn install}
    \item Výsledný adresář {\tt target/module/org} nahrajte do adresáře {\tt modules/system/la\linebreak yers/base/} v adresáři WildFly.
    \item Spusťe konzolu WildFly, {\tt bin/jboss-cli.sh} a následujícím příkazem nainstalujte a přidejte subsystém. Pro {\it standalone} režim:
      \begin{lstlisting}
/extension=org.picketbox.jsmpolicy.subsystem:add
/subsystem=jsmpolicy:add
      \end{lstlisting}
      Pro profil {\tt full} v režimu {\tt domain}:
      \begin{lstlisting}
/extension=org.picketbox.jsmpolicy.subsystem:add
/profile=full/subsystem=jsmpolicy:add
      \end{lstlisting}
    \item Jestliže se oba příkazy zdaří, subsystém {\tt jsmpolicy} byl úspěšně nainstalován a přidán do profilu. (V příkladu pro {\tt domain} režim do profilu {\tt full}.)
  \end{enumerate}
  
  \item Proveďte instalaci rozšíření webové konzoly WildFly {\tt jsm-policy-console-hal}:
  \begin{enumerate}
    \item Nakopírujte do zapisovatelného adresáře repozitář {\tt jsm-policy-console-hal}. Získat jej můžete z přiloženého CD nebo z GitHub:
      \newline\url{https://github.com/honza889/jsm-policy-console-hal}
    \item Proveďte kompilaci za pomoci nástroje Maven: {\tt mvn install}
    \item Výsledný archiv ({\tt build/app/target/jboss-as-console-*-resources.jar}) nahrajte do adresáře {\tt modules/system/layers/base/org/jboss/as/console/main} v adresáři WildFly.
    \item Upravte soubor {\tt module.xml}, aby atribut {\tt path} elementu {\tt resource-root} odpovídal názvu souboru zkopírovaném v minulém kroku.
    \item Proveďte restart aplikačního serveru WildFly a následně znovunačtěte webovou stránku s webovou konzolou WildFly. (Např. stisknutím klávesy F5 nebo Ctrl+F5 ve webovém prohlížeči.)
    \item Jestliže se ve webové administrační konzole v nabídce subsystémů {\it (Subsystems)} ukáže nová položka {\bf JSM Policy}, bylo toto rozšíření úspěšně nainstalováno.
  \end{enumerate}
  
\end{enumerate}


\section{Možné příčiny problémů}
\begin{enumerate}
    \item V nabídce subsystémů ve webové konzoli se nezobrazuje subsystém JSM Policy.
    \begin{enumerate}
        \item Ve webové konzole je vybraný profil, v kterém není nainstalován subsystém {\tt jsmpolicy} (dle bodu 3 postupu). (V případě {\it standalone} režimu: Není nainstalován subsystém {\tt jsmpolicy}.)
        \item Zkompilované rozšíření webové konzoly nebylo (dle bodu 4c postupu) nakopírováno do správného adresáře.
        \item V tomto adresáři nebyl upraven soubor {\tt module.xml} tak, aby odkazoval na soubor nakopírovaného rozšíření. (dle bodu 4d)
        \item Webová stránka s webovou konzolí WildFly nebyla znovunačtena nebo server WildFly restartován (dle bodu 4e postupu).
    \end{enumerate}
\end{enumerate}


\section{Spuštění integračních testů}

Jestliže jste úspěšně dokončili instalaci rozšíření aplikačního, můžete provést instalaci a spuštění integračních testů popsaných v kapitole \ref{testovani}.

\begin{enumerate}
  \item Nakopírujte do zapisovatelného adresáře repozitář {\tt jsm-policy-test}. Získat jej můžete z přiloženého CD nebo z GitHub:
    \newline\url{https://github.com/honza889/jsm-policy-test}
  \item Ujistěte se že testovací servery testovací domény WildFly s nainstalovanými rozšířeními jsou spuštěné. Integrační test využívá dvou testovacích serverů.
  \item Upravte v konfiguračním souboru nástroje Maven {\tt agent/pom.xml} údaje nezbytné pro přihlášení k testovací doméně.
  \item Upravte v souboru {\tt manager/src/test/java/org/picketbox/jsmpolicy/test/Con stants.java} informace o testovacích serverech a testovací doméně.
  \item Spusťte skript {\tt ./test.sh}
  \item Výsledky testů budou uloženy ve formě HTML dokumentu do souboru {\tt manager/targe t/site/surefire-report.html}, který by měl být po skončení testů automaticky zobrazen.
\end{enumerate}

%\chapter{Manual}
%\chapter{Konfigrační soubor}
%\chapter{RelaxNG Schéma konfiguračního soboru}
%\chapter{Plakat}

