%============================================================================
% tento soubor pouzijte jako zaklad
% (c) 2008 Michal Bidlo
% E-mail: bidlom AT fit vutbr cz
%============================================================================
% kodovaní: iso-8859-2 (zmena prikazem iconv, recode nebo cstocs)
%----------------------------------------------------------------------------
% zpracování: make, make pdf, make desky, make clean
% připomínky posílejte na e-mail: bidlom AT fit.vutbr.cz
% vim: set syntax=tex encoding=latin2:
%============================================================================

\documentclass[cover]{fitthesis} % odevzdani do wisu - odkazy, na ktere se da klikat
%\documentclass[cover,print]{fitthesis} % pro tisk - na odkazy se neda klikat

%\documentclass[english,print]{fitthesis} % pro tisk - na odkazy se neda klikat
%      \documentclass[english]{fitthesis}
% * Je-li prace psana v anglickem jazyce, je zapotrebi u tridy pouzit 
%   parametr english nasledovne:
%      \documentclass[english]{fitthesis}
% * Neprejete-li si vysazet na prvni strane dokumentu desky, zruste 
%   parametr cover

% zde zvolime kodovani, ve kterem je napsan text prace
% "latin2" pro iso8859-2 nebo "cp1250" pro windows-1250, "utf8" pro "utf-8"
%\usepackage{ucs}
\usepackage[utf8]{inputenc}
\usepackage[T1, IL2]{fontenc}
\usepackage{url}
\DeclareUrlCommand\url{\def\UrlLeft{<}\def\UrlRight{>} \urlstyle{tt}}

%zde muzeme vlozit vlastni balicky
\usepackage{subcaption}
\DeclareUnicodeCharacter{00A0}{~} % oprava chyby Unicode char \u8:�. not set up for use with LaTeX

\usepackage{color}
\usepackage{listings}
\definecolor{light-gray}{gray}{0.95}
\definecolor{dark-gray}{gray}{0.8}
\renewcommand{\lstlistingname}{Ukázka kódu}
\lstset{
language=Java,                          % Code langugage
basicstyle=\ttfamily,                   % Code font, Examples: \footnotesize, \ttfamily
stepnumber=1,                           % Step between two line-numbers
numbersep=5pt,                          % How far are line-numbers from code
%frame=l,                               % A frame around the code (lrtb/none)
frame=lrtb,                             % A frame around the code (lrtb/none)
tabsize=2,                              % Default tab size
captionpos=b,                           % Caption-position = bottom
breaklines=true,                        % Automatic line breaking?
breakatwhitespace=false,                % Automatic breaks only at whitespace?
showspaces=false,                       % Dont make spaces visible
showtabs=false,                         % Dont make tabls visible
columns=flexible,                       % Column format
extendedchars=true,                     % lets you use non-ASCII characters; for 8-bits encodings only, does not work with UTF-8
showstringspaces=false,
%backgroundcolor=\color{light-gray},
rulecolor=\color{dark-gray},
xleftmargin=5pt,
xrightmargin=5pt,
}

% =======================================================================
% balíček "hyperref" vytváří klikací odkazy v pdf, pokud tedy použijeme pdflatex
% problém je, že balíček hyperref musí být uveden jako poslední, takže nemůže
% být v šabloně
\ifWis
\ifx\pdfoutput\undefined % nejedeme pod pdflatexem
\else
  \usepackage{color}
  \usepackage[unicode,colorlinks,hyperindex,plainpages=false,pdftex]{hyperref}
  \definecolor{links}{rgb}{0.4,0.5,0}
  \definecolor{anchors}{rgb}{1,0,0}
  \def\AnchorColor{anchors}
  \def\LinkColor{links}
  \def\pdfBorderAttrs{/Border [0 0 0] }  % bez okrajů kolem odkazů
  \pdfcompresslevel=9
\fi
\fi

%Informace o praci/projektu
%---------------------------------------------------------------------------
\projectinfo{
  %Prace
  project=BP,            %typ prace BP/SP/DP/DR
  year=2014,             %rok
  date=\today,           %datum odevzdani
  %Nazev prace
  title.cs=Centralizace správy bezpečnostních politik v Javě,  %nazev prace v cestine
  title.en=Java Security Policy Centralization, %nazev prace v anglictine
  %Autor
  author=Jan Kalina,   %jmeno prijmeni autora
  %author.title.p=Bc., %titul pred jmenem (nepovinne)
  %author.title.a=PhD, %titul za jmenem (nepovinne)
  %Ustav
  department=VCIT, % doplnte prislusnou zkratku: UPSY/UIFS/UITS/UPGM
  %Skolitel
  supervisor=Zdeněk Letko, %jmeno prijmeni skolitele
  supervisor.title.p=Ing., %titul pred jmenem (nepovinne)
  supervisor.title.a=Ph.D., %titul za jmenem (nepovinne)
  %Klicova slova, abstrakty, prohlaseni a podekovani je mozne definovat 
  %bud pomoci nasledujicich parametru nebo pomoci vyhrazenych maker (viz dale)
  %===========================================================================
  %Klicova slova
  keywords.cs={Klíčová slova v českém jazyce.}, %klicova slova v ceskem jazyce
  keywords.en={Klíčová slova v anglickém jazyce.}, %klicova slova v anglickem jazyce
  %Abstract
  abstract.cs={Výtah (abstrakt) práce v českém jazyce.}, % abstrakt v ceskem jazyce
  abstract.en={Výtah (abstrakt) práce v anglickém jazyce.}, % abstrakt v anglickem jazyce
  %Prohlaseni
  declaration={Prohlašuji, že jsem tuto bakalářskou práci vypracoval samostatně pod vedením pana ...},
  %Podekovani (nepovinne)
  acknowledgment={Zde je možné uvést poděkování vedoucímu práce a těm, kteří poskytli odbornou pomoc.} % nepovinne
}

%Abstrakt (cesky, anglicky)
\abstract[cs]{
  WildFly je platformou pro distribuované prostředí splňující specifikaci Java Enterprise Edition.
  Tato práce se zabývá možnostmi centrální správy bezpečnostních politik v tomto prostředí.
  Bezpečnostní politika je sada oprávnění, na které virtuální stroje Javy (JVM) omezují možnosti spuštěných aplikací.
  Právě možnosti používání bezpečnostních politik byly ve WildFly dosud silně omezeny.
  Výsledkem práce jsou rozšiřující moduly WildFly, doplňující programové rozhraní WildFly a webovou administrační konzolu WildFly o možnost centrálního
  nasazování bezpečnostních politik na jednotlivé servery domény WildFly bez potřeby jejich restartu.
  Součástí výsledku je také záplata samotného jádra WildFly, řešící problém výměny bezpečnostní politiky za běhu JVM.
}
\abstract[en]{
  WildFly is a platform for distributed environment which meets specification of Java Enterprise Edition.
  This thesis deals with possibilities of centralized management of security policies in this environment.
  Security policy is a set of permissions to which the Java virtual machine (JVM) limits possibilities of running applications.
  Just possibilities of security policy using was in WildFly so far heavily restricted.
  The result of the thesis are extensions of WildFly which add possibility of central deployment of security policies to individual servers,
  without need for restart that server, into program interface of WildFly and into WildFly management console.
  Part of result is also patch of core of WildFly, which solve problem of exchange security policy at runtime of JVM.
}

%Klicova slova (cesky, anglicky)
\keywords[cs]{
  JBoss, WildFly, Java, bezpečnostní politika, oprávnění, správce bezpečnosti, ochranná doména.
}
\keywords[en]{
  JBoss, WildFly, Java, security policy, permission, security manager, protection domain.
}

%Prohlaseni
\declaration{
  Prohlašuji, že jsem tuto bakalářskou práci vypracoval samostatně pod vedením pana Ing. Zdeňka Letka, Ph.D.
  Další informace mi poskytl odborný konzultant Ing. Peter Škopek.
  Uvedl jsem všechny literární prameny a publikace, ze kterých jsem čerpal.
}

%Podekovani (nepovinne)
\acknowledgment{
  Zde bych rád poděkoval svému vedoucímu práce Ing. Zdeňku Letkovi a svému odbornému konzultantovi
  Ing. Petru Škopkovi za jejich čas, trpělivost, ochotu a cenné rady.
}

\begin{document}
  \maketitle % titulni strany
  \tableofcontents % obsah
  
  %\listoffigures % seznam obrazku
  %\listoftables % seznam tabulek
  
  %%%%%%%%%%%%%%%%%%%%%%%%%%%%%%%%%%%%%%%%%%%%%%%%%%%%%%%%%%%%%%%%%%%%%%%%%%%%%%
% Jan Kalina <xkalin03@stud.fit.vutbr.cz> 2013
%%%%%%%%%%%%%%%%%%%%%%%%%%%%%%%%%%%%%%%%%%%%%%%%%%%%%%%%%%%%%%%%%%%%%%%%%%%%%%

%%%%%%%%%%%%%%%%%%%%%%%%%%%%%%%%%%%%%%%%%%%%%%%%%%%%%%%%%%%%%%%%%%%%%%%%%%%%%%
\chapter{Bezpečnost v Javě (Teoretický úvod)}
%%%%%%%%%%%%%%%%%%%%%%%%%%%%%%%%%%%%%%%%%%%%%%%%%%%%%%%%%%%%%%%%%%%%%%%%%%%%%%

%=============================================================================
\section{Bezpečnost}
%=============================================================================

Protože se tato práce zabývá bezpečností, což je pojem, který lze v různých souvislostech chápat zcela odlišně, je nanejvýš vhodné nejprve specifikovat co termínem bezpečnosti míníme a z kterého pohledu se jí budeme zabývat.

Scott Oaks definuje ve své knize Java Security bezpečnost jako souhrn následujících kritérií: \cite[1.1]{oaks}

\begin{itemize}
  \item Bezpečí vůči zákeřnému software - programy by neměly být schopny poškodit prostředí hostitelského počítače.
  \item Bezpečí vůči Velkému bratru - programům by mělo být bráněno ve šmírování uživatele -- programy by neměly být schopny svévolně číst soukromé informace na počítači na kterém běží, ani na síti ke které je tento počítač připojen.
  \item Autentizace - identita autorů programu by měla být ověřována (typicky pomocí elektronickému podpisu).
  \item Šifrování - data odesílaná a přijímaná programem by měla být šifrována.
  \item Auditovatelnost - potenciálně škodlivé operace by měly být vždy zaznamenávány.
  \item Specifikovanost - program by měla doprovázet specifikace bezpečnostních pravidel, které program dodržuje.
  \item Verifikovanost - pro prováděné operace by měla být stanovena pravidla, proti kterým by měly být verifikovány.
  \item Dbaní na dobré vychování - programům by mělo být bráněno v užívání příliš mnoha systémových prostředků.
\end{itemize}

V souvislosti s víceuživatelskými prostředími se pak pojem bezpečnosti vyskytuje ještě v další rovině, kdy nejde o bezpečí před běžícím programem, ale o způsob jakým může naopak program ověřit, kdo je jeho uživatelem a zdali má právo po něm žádat vykonání operace, o jejíž vykonání žádá.

V této práci se však budeme zabývat bezpečností ve smyslu prvních tří uvedených kritérií, tedy ve smyslu ochrany prostředí a dat hostitele před programy.

%=============================================================================
\section{Java}
%=============================================================================

Programy v jazyce Java bývají překládány do platformě nezávislého a efektivněji než kód v jazyce Java interpretovatelného mezikódu, takzvaného bytekódu.
Bytekód bývá zpravidla interpretován virtuálním strojem Javy (JVM - Java Virtual Machine).
JVM je abstraktní výpočetní stroj. Podobně jako reálné výpočetní stroje má svoji instrukční sadu a paměť se kterou může manipulovat, ale na rozdíl od nich pro něj neexistuje jeho fyzická implementace, pouze emulovaná implementace softwarová.
To znamená že její kód není prováděn nativně hardwarem, ale je interpretován speciálním programem, interpreterem, který je sám zkompilován do nativního kódu dané platformy.
To mimo nezávislosti na platformě přináší také vyšší stupeň abstrakce.

Díky tomu, že programy nepřistupují ani nemohou přistupovat ke zdrojům fyzického stroje přímo, ale pouze zprostředkovaně skrze zdroje virtuální stroje Javy, si není těžké představit, že by použití takovéhoto virtuálního stroje mohlo mít i významný bezpečnostní efekt.
Protože programy v JVM mohou k fyzickým zdrojům počítače (např. k souborům nebo k síti) přistupovat jen skrze JVM, zablokování takového přístupu ze strany JVM není nikterak složité - JVM stačí odmítnout takový požadavek a interpretovaný program nemá možnost JVM obejít.

%=============================================================================
\section{Java Security Manager}
%=============================================================================

Pro maximální nastavitelnost omezení kladených na programy běžících na virtuáních strojích Javy nerozhoduje o povolení nebo zablokování operace nad zdrojem JVM samotná JVM, ale dotazuje se speciálního objektu třídy {\tt java.lang.SecurityManager}, nebo jeho podtřídy.

Podtřída je třída dědící atributy a metody (operace, přijímané zprávy) své nadtřídy a reference na její instanci může být vložena do proměnné určené pro referenci na její nadtřídu. Jakákoli podtřída třídy SecurityManager tedy bude vždy přijímat všechny zprávy, které přijimá třída SecurityManager, přičemž ty, které nebude sama implementovat, budou přejaty z její nadtřídy SecurityManager.

Security manager, který JVM použije při svém startu lze ovlivnit skrze konfigurační proměnnou JVM -- {\tt java.security.manager}. Za běhu je možné aktuálně používaný Security manager zjistit voláním {\tt System.getSecurityManager() } a vypnout nebo vyměnit voláním {\tt System.setSecurityManager()}. Volání těchto metod bývá samo chráněno Security managerem, takže nehrozí že by se program neoprávněně zbavil omezení, které na něj uvalil Security manager vypnutím nebo vyměněním Security manageru.

Hodnotu konfigurační proměnné {\tt java.security.manager} po startu JVM je možné nastavit za pomoci k tomu určeného parametru {\tt -D}. Pro spuštění programu s výchozím Security managerem můžeme tedy použít příkaz:

\begin{verbatim}
java -Djava.security.manager=default ProgramABC
\end{verbatim}

Základní možné hodnoty proměnné {\tt java.security.manager} a jim odpovídající třídy objektů security managera popisuje následující tabulka.

\begin{center}
    \begin{tabular}{| l | l |}
    \hline
    Parametr příkazu java & Použitý JSM \\ \hline
    (parametr nepoužit)                                      & {\tt null                      } \\
    {\tt -Djava.security.manager                           } & {\tt java.lang.SecurityManager } \\
    {\tt -Djava.security.manager=default                   } & {\tt java.lang.SecurityManager } \\
    {\tt -Djava.security.manager=java.lang.SecurityManager } & {\tt java.lang.SecurityManager } \\
    {\tt -Djava.security.manager=TestovaciSM               } & {\tt TestovaciSM               } \\
    \hline
    \end{tabular}
\end{center}

Poslední řádek demonstruje použití vlastní třídy objektu security managera. Způsob vytvoření vlastního Security manageru bude podrobněji rozebrán v další kapitole. Jestliže je zde uvedena neexistující třída, skončí inicializace JVM vyjímkou a vykonávání programu nebude vůbec zahájeno:

\begin{verbatim}
Error occurred during initialization of VM
java.lang.InternalError: Could not create SecurityManager: neexistujici.SM
    at sun.misc.Launcher.<init>(Launcher.java:106)
    at sun.misc.Launcher.<clinit>(Launcher.java:57)
    at java.lang.ClassLoader.initSystemClassLoader(ClassLoader.java:1486)
    at java.lang.ClassLoader.getSystemClassLoader(ClassLoader.java:1468)
\end{verbatim}

Tím že je zabráněno startu JVM v případě chybné konfigurace Security manageru -- je tedy uplatněn bezpečnostní princip, podle kterého musí systém zůstat bezpečný i v případě poruchového stavu.

%=============================================================================
\section{Implementace vlastního Security manageru}
%=============================================================================

V této podkapitole bude vysvětlen způsob stanovení bezpečnostní politiky vytvořením vlastního Security manageru.
Security manager je objekt, na který JVM při požadavku Java API deleguje rozhodnutí o povolení nebo nepovolení operace nad zdrojem.
Tento objekt je instancí třídy {\tt java.lang.SecurityManager} nebo její podtřídy.
Algoritmus používaný Java API pro rozhodnutí o vykonání nebo nevykonání potenciálně nebezpečné operace je následující: \cite[4.1.1]{oaks}

\begin{enumerate}
  \item Uživatelská aplikace žádá Java API o provedení dané operace
  \item Java API se dotáže Security manageru zda operaci povolit zavoláním patřičné metody Security manageru.
  \item Nemá-li být operace povolena, vrátí Security manager vyjímku {\tt SecurityException}, která je předána uživatelské aplikaci, čímž je vykonání operace zabráněno.
  \item Není-li vyhozena žádná vyjímka, operace se považuje za povolenou a je provedena.
\end{enumerate}

Vlastní Security manager tedy vytvoříme rozšířením třídy {\tt SecurityManager}, přičemž přepíšeme metody rozhodující o povolení akce, kterou chceme povolit, nebo pro kterou chceme stanovit vlastní algoritmus rozhodující zda akci povolit nebo ne.
Protože standardní Security manager implicitně všechny operace zakazuje, stačí nám implementovat metody autorizující provedení operací, které chceme povolit a operacemi které nikdy povolit chtít nebudeme se nebudeme muset zabývat.

Příklad níže ukazuje jednoduchý Security manager, který o povolení každé akce rozhodne na základě interakce s uživatelem. Při pokusu programu o provedení potenciálně nebezpečné operace je uživateli zobrazen dotaz, zda chce tuto operaci povolit nebo ne. V případě negativní odpovědi je jejímu provedení vrácením vyjímky {\tt SecurityException} zabráněno.

\begin{verbatim}
public class TestovaciSM extends SecurityManager {
    @Override
    public void checkPermission(Permission permission) {
        
        System.out.println("Povolit " + permission.toString() + " ?");
        if(!askUser()){ // jestliže uživatel nezvolí "ano"
            throw new SecurityException("Operace byla zakázána uživatelem");
        }
        
    }
}
\end{verbatim}

Takto vytvořený Security manager pak můžeme na vlastní program nechat použít za pomoci dříve zmíněného příkazu:

\begin{verbatim}
System.setSecurityManager(new TestovaciSM());
\end{verbatim}

Jinak (pro libovolný program) můžeme použití tohoto Security manageru nastavit při startu JVM:

\begin{verbatim}
java -Djava.security.manager=TestovaciSM HelloWorld
\end{verbatim}

Nutno dodat že tato konfigurační proměnná je uplatněna při startu JVM, její změnou za běhu JVM nebude použití Security manageru ovlivněno.

Zde uvedený Security manager je opravdu pouze demonstrativní, v čemž se můžeme utvrdit, spustíme-li jej nad jednoduchou aplikací Hello world. Množství dotazů kladených na Security manager je i u takto primitivní aplikace příliš velké, než aby je bylo možné nechat zodpovědět uživatelem.

Stanovení bezpečnostní politiky vytvořením Security manageru ale nepochybně možné je. Vyžaduje ale znalosti programování v Javě od správce počítače na němž JVM běží a od Javy verze 2 se již pro stanovení bezpečnostní politiky nepoužívá.

%=============================================================================
\section{Zavaděče tříd (Classloader)}
%=============================================================================

Než se dostaneme k v současnosti využívanému způsobu stanovování bezpečnostní politiky, pozastavíme se nad tématem zavádění tříd.
Ačkoli se na první pohled může zdát, že toto téma s bezpečností příliš nesouvisí, opak je pravdou.
Jen a pouze zavaděč tříd, který třídu do paměti zavádí totiž může určit původ třídy, od kterého se oprávnění kódu této třídy odvozuje.

Zavaděč je objekt odpovědný za načítání tříd a rozhraní v Javě. Na základně binárního názvu třídy (tedy názvu používaného v bytekódu, např. {\tt java.lang.String} nebo {\tt java.security.KeyStore\$Builder\$FileBuilder\$1}) se pokusí vyhledat a načíst třídu daného názvu. \cite{refClassLoader}

Třída každého zavaděče musí být podtřídou třídy {\tt java.lang.ClassLoader} a musí implementovat metodu {\tt findClass()}, která provádí právě samotné vyhledání třídy podle názvu. Jejím výstupem je objekt třídy {\tt Class} představující samotnou třídu. Při programování běžných aplikací můžeme na objekty této třídy narazit například při snaze získat název třídy neznámého objektu jako řetězec za běhu aplikace: \cite{refClassLoader}

\begin{verbatim}
Class tridaObjektuX = x.getClass();
String nazevTridyObjektuX = c.getName();
\end{verbatim}

Jako původ třídy ({\tt CodeSource}) souhrně označujeme URL adresu, ze které byla třída získána, a elektronické podpisy, kterými přitom byl její JAR archiv opatřen.
Na základě původu je třída zařazena do ochranné domény. Mají-li dvě třídy stejný původ, je jim přiřazena stejná ochranná doména. \cite[5.1]{oaks}\cite{sourceSecureClassLoader}

Ochranná doména ({\tt ProtectionDomain}) je uskupení zdrojových kódů a oprávnění, které jsou tomuto kódu udělena.
Speciálním případem je tzv. systémová ochranná doména, do které spadají třídy zavedení interním zavaděčem tříd (viz. \ref{interniZavadec}), jejiž reference na ochrannou doménu je nastavena na {\tt null} a jejich oprávnění nejsou omezené. \cite[5.4]{oaks}

%-----------------------------------------------------------------------------
\subsection{Interní zavaděč tříd} \label{interniZavadec}
%-----------------------------------------------------------------------------

Interní zavaděč tříd je zavaděč používaný pro načítání tříd samotného Java API a zavaděčů uživatelských tříd.
Interní zavaděč je součástí JVM. Je napsán převážně v nativním kódu a k načítání tříd využívá nativní metody pro přístup k souborovému systému operačního systému.
Interní zavaděč načítá třídy ze souborů, jejiž adresu odvozuje z názvu třídy, balíčku a proměnné prostředí CLASSPATH. \cite[3.2.1]{oaks}

Je-li tedy JVM spuštěna následujícím příkazem:

\begin{verbatim}
java -classpath /srv/classes muj.program.Test
\end{verbatim}

Bude spuštěna metoda {\tt main()} třídy {\tt Test}, jež bude hledána v souboru:

\begin{verbatim}
/srv/classes/muj/program/Test.class
\end{verbatim}

O třídách načtených interním zavaděčem se často mluví jako o třídách bez zavaděče, protože jejich objekt {\tt Class} má referenci na zavaděč nastavenu na {\tt null}. \cite[3.2.1]{oaks} Zároveň je {\tt null}ová také ochranná doména takto načtených tříd. Třídy načtené interním zavaděčem tak nemají svá oprávnění omezena.

Interní zavaděč je v současnosti používán k načítání jen nejzákladnějších tříd, zejména tříd Java API, tedy tříd používajících součásti napsané v nativním kódu. Ostatní třídy z {\tt CLASSPATH} jsou obvykle načítány Zavaděčem tříd z URL.

%-----------------------------------------------------------------------------
\subsection{Zavaděč tříd z URL}
%-----------------------------------------------------------------------------

Zavaděč tříd z URL ({\tt URLClassLoader}) hledá třídy také na URL adresách určených polem objektů {\tt URL}, předaných jeho kontruktoru. Je jedním z nejpoužívanějších zavaděčů tříd v Javě -- používá se i k vyhledávání méně základních tříd z {\tt CLASSPATH} a dokáže třídy načítat nejen z adresářů, ale i z JAR archivů, je-li jako adresa uvedena adresa JAR archivu. \cite[3.2.5]{oaks}

\begin{verbatim}
URLClassLoader urlClassLoader = new URLClassLoader(new URL[]{
    new URL("http://server/directory/"),
    new URL("file:/srv/classes/")
}, parentClassloader);
\end{verbatim}

Zavaděč tříd z URL je jedním z bezpečných zavaděčů ({\tt SecureClassLoader}). To znamená že automaticky nastavuje ochranou doménu třídám, které načítá.
Ochranná doména která bude zvolena závisí na následujících parametrech: \cite{refPolicyFiles}

\begin{itemize}
  \item {\tt codeBase}: URL adresa ze které byla třída načtena.
  \item {\tt signedBy}: Alias podpisu, kterým byl JAR archiv z něhož byla třída načtena podepsán.
  \item {\tt principal}: Operaci žádá kód spuštěný uživatelem s oprávněním uvedeným v pravidlu bezpečnostní politiky.
\end{itemize}

Třídy, které mají všechny tyto parametry společné, patří do stejné ochranné domény a mají tak stejná oprávnění.

%=============================================================================
\section{Access Controller a soubory bezpečnostní politiky}
%=============================================================================

Access controller je mechanismus využívající ochranné domény tříd, který standardní Security manager (od Javy verze 2) využívá pro rozhodnutí, zda operaci požadovanou uživatelským programem povolit či nikoliv. Aby tedy byla aplikována omezení stanovená Access controller, musí být JVM spuštěna se standardním Security managerem. \cite[5]{oaks}

Access controller, stejně jako Security manager, umožňuje omezit, zdali může být operace nad zdrojem provedena nebo ne. Zdroji zda však již nejsou jen zdroje samotné JVM. Access controller mohou využívat i uživatelské procesy pro omezení přístupu ke zdrojům, které poskytují. \cite[5]{oaks}

Máme-li tedy knihovní třídu zprostředkovávající přístup k databázi v souboru po záznamech, při použití oprávnění jen na úrovni Java API by bylo možné programu přidělovat oprávnění jen k celému souboru. Access controller ale umožňuje považovat jednotlivé záznamy databáze v souboru za samostatné zdroje, čímž umožňuje různým programům přidělit oprávnění jen k vybraným záznamům v databázovém souboru.

Access controller je schopný pravidla bezpečnostní politiky načítat z textového konfiguračního souboru -- tzv. souboru bezpečnostní politiky. Vzniká tak mnohem snazší způsob stanovení bezpečnostní politiky, kterou budou programy v JVM omezeny -- namísto tvorby vlastní třídy Security manageru bude pro stanovení bezpečnostní politiky stačit upravovat obsah souboru bezpečnostní politiky. \cite[5]{oaks}

Ochraná doména (Protection Domain) je uskupení zdrojových kódů a oprávnění, které jsou tomuto kódu udělena. Každá třída spadá vždy do jedné ochrané domény, která jí je přidělena zavaděčem při jejím načítání. Jen zavaděč totiž zná původ třídy, na základě kterého jsou třídy ochraným doménám přidělovány. Speciálním případem jsou třídy zavedené interním zavaděčem (viz. \ref{interniZavadec}), které spadají do systémové ochrané domény a mají tak vždy povoleno vše. \cite[5.4]{oaks}

Access controller rozhoduje o povolení nebo nepovolení operace na základně průniku oprávnění ochraných domén tříd metod, jež vedly k zavolání metody provádějící citlivou operaci a dotazující se na její legálnost Access controlleru, ať už přímo nebo skrze Security manager.

%-----------------------------------------------------------------------------
\subsection{Implementace Access Controlleru}\label{implementaceAC}
%-----------------------------------------------------------------------------

Access controller je třídou {\tt java.security.AccessController} a voláním jeho statických metod na něj standardní security manager deleguje rozhodnutí o legálnosti provedení potenciálně nežádoucí operace uživatelského programu. Konkrétně kontrolu obstarává metoda {\tt checkPermission(Permission)}, která v případě neoprávněnosti požadavku na provedení operace způsobí vyhození vyjímky {\tt AccessControlException}. Jejím vstupem je objekt třídy {\tt Permission}, která reprezentuje oprávnění, jehož vlastnictví je testováno. \cite[5.5]{oaks}\cite[6]{oaks}

Následující příklad ukazuje způsob, jakým se může kód dotázat Access controlleru, zdali má zadané oprávnění: \cite[5.5]{oaks}

\begin{verbatim}
try {
    // má tento program oprávnění připojit se na lokální port 80?
    AccessController.checkPermission(
        new SocketPermission("localhost:80", "connect")
    );
    System.out.println("Program má oprávnění připojit se na port");
} catch (AccessControlException ace) {
    System.out.println("Program nemá oprávnění připojit se na port");
}
\end{verbatim}

Access controller rozhoduje o povolení nebo nepovolení operace na základně průniku oprávnění ochraných domén tříd metod, jež vedly k zavolání metody {\tt checkPermission()}. Konkrétně jde o metody nacházející se během volání {\tt checkPermission()} na zásobníku volání (Stack trace). Volala-li tedy metoda {\tt main} třídy {\tt Main} metodu {\tt secured} třídy {\tt Secured}, která následně zavolala metodu {\tt checkPermission()}, musí mít požadované oprávnění přiděleny ochrané domény tříd {\tt Main} i {\tt Secured}, aby toto volání neskončilo vyjímkou vyjadřující zamítnutí provedení operace ze strany Access controlleru. \cite[5.5]{oaks}\cite[6.1]{oaks}


%Access controller je založen na čtyř konceptech: \cite[5]{oaks}
%
%\begin{itemize}
%  \item Zdroje kódu (CodeSource): Objekt představující původ třídy -- adresu URL z které byla získána a certifikáty (elektronické podpisy), kterými byla opatřena %\cite[5.1]{oaks}
%  \item Oprávnění (Permission): Objekt představující operaci o jejíž povolení se rozhoduje, konkrétně se skládá z trojice: typ oprávnění, název zdroje, povolené akce \cite[5.2]{oaks}
%  \item Politiky (Policy): Objekt představující souhrn oprávnění udělených danému zdrojovému kódu \cite[5.3]{oaks}
%  \item Ochranná doména (ProtectionDomain): Objekt představující zdrojové kódy a jejich oprávnění \cite[5.4]{oaks}
%\end{itemize}

%-----------------------------------------------------------------------------
\subsection{Privilegované bloky kódu}
%-----------------------------------------------------------------------------

Vraťme se nyní k našemu příkladu, kdy chceme různým programům přidělovat oprávnění k jednotlivým databázovým záznamům jako k samostatným zdrojům systému, ale nechceme jim povolit přístup k samotnému databázovému souboru.

Ze způsobu fungování Access controlleru tak jak jsme si jej zatím popsali totiž vyplývá, že aby mohla knihovna zprostředkovávající přístup k databázi přistupovat k databázovému souboru, musí mít oprávnění pracovat s databázovým souborem jak samotná databázová knihovna, tak i každá třída podílející se na jejím volání.

Tento stav je samozřejmě nepřípustný, protože oprávnění na úrovni uživatelského kódu by tímto ztratila význam -- buď by k databázi nemohla přistupovat knihovna volaná kódem, jež má oprávnění jen k některým záznamům, nebo by naopak měl jakýkoli kód, který by potřeboval přistupovat k záznamům v databázi, oprávnění k celému databázovému souboru, čímž by mohl jakákoli omezení na úrovni knihovny zprostředkující přístup k databázi obejít.

Právě tento problém řeší privilegované bloky kódu. Kód v privilegovaném bloku je spuštěn se samostatným zásobníkem volání. Při ověřování oprávnění při přístupu k jakémukoli zdroji tak ověřování skončí u třídy s tímto blokem.

{\bf Část kódu knihovny, jež vložíme do privilegovaného bloku, bude tedy spuštěna s oprávněními této knihovny bez ohledu na oprávnění kódu, jež metodu knihovny zavolal.}

%,,,,,,,,,,,,,,,,,,,,,,,,,,,,,,,,,,,,,,,,,,,,,,,,,,,,,,,,,,,,,,,,,,,,,,,,,,,,,
\subsubsection{Příklad použití privilegovaného bloku}
%'''''''''''''''''''''''''''''''''''''''''''''''''''''''''''''''''''''''''''''

Vykonávání privilegovaného bloku je implementováno jako nativní metoda Access controlleru, které je předán objekt (pod)třídy {\tt PrivilegedAction}, jehož metoda {\tt run()} je spuštěna privilegovaně.
Tento příklad ukazuje jednoduchý příklad použití privilegovaného bloku: \cite{refAccessController}

\begin{verbatim}
class DatabazeVSouboru {
  public void provedOperaciNadDatabazi(){
    // Kód zde bude omezen oprávněními volající třídy
    AccessController.doPrivileged(new PrivilegedAction<Void>() {
      public Void run() {
        // Kód zde bude spuštěn nezávisle na oprávnění volající třídy
      }
    }
  }
}
\end{verbatim}

%,,,,,,,,,,,,,,,,,,,,,,,,,,,,,,,,,,,,,,,,,,,,,,,,,,,,,,,,,,,,,,,,,,,,,,,,,,,,,
\subsubsection{Příklad s knihovnou pro přístup k databázi}
%'''''''''''''''''''''''''''''''''''''''''''''''''''''''''''''''''''''''''''''

\begin{verbatim}
class DatabazeVSouboru {
  public Zaznam nactiZaznam(String klic){
    
    // mimo privilegovaný blok ověříme, že volající kód má oprávnění
    // k záznamu o který žádá
    AccessController.checkPermission(new ZaznamPermission(klic, "nacteni"));
    
    // protože oprávnění má, bez ohledu na to že volající kód nemá
    // oprávnění přistupovat k databázovému souboru záznam načteme
    AccessController.doPrivileged(new PrivilegedAction<Zaznam>() {
      public Zaznam run() {
        // Zde bude práce s databázovým souborem
        
      }
    }
  }
}
\end{verbatim}

%-----------------------------------------------------------------------------
\subsection{Soubory bezpečnostní politiky}
%-----------------------------------------------------------------------------

Rozdíl mezi - a * na konci adresy atd...

Povolované operace jsou definovány za pomoci tří kritérií: \cite{jdkdocPolicyFiles}

\begin{itemize}
  \item {\tt permission\_class\_name}: Typ oprávnění -- název podtřídy třídy {\tt Permission}.
  \item {\tt target\_name}: Název zdroje -- souboru, serveru a portu apod.
  \item {\tt action}: Název povolované operace, např. {\tt read} pro čtení a {\tt delete} pro smazání souboru.
\end{itemize}

Způsob použití bude upřesněn v kapitole Soubory bezpečnostní politiky.





\begin{verbatim}
http://docs.oracle.com/javase/7/docs/api/java/lang/ClassLoader.html
http://docs.oracle.com/javase/7/docs/api/java/lang/Class.html
http://docs.oracle.com/javase/jndi/tutorial/beyond/misc/classloader.html
http://www.linuxsoft.cz/article.php?id_article=1123
\end{verbatim}












%%%%%%%%%%%%%%%%%%%%%%%%%%%%%%%%%%%%%%%%%%%%%%%%%%%%%%%%%%%%%%%%%%%%%%%%%%%%%%
\chapter{Závěr}
%%%%%%%%%%%%%%%%%%%%%%%%%%%%%%%%%%%%%%%%%%%%%%%%%%%%%%%%%%%%%%%%%%%%%%%%%%%%%%

 % viz. obsah.tex
  
  % Pouzita literatura
  % ----------------------------------------------
\ifczech
  \bibliographystyle{czechiso}
\else
  \bibliographystyle{plain}
%  \bibliographystyle{alpha}
\fi
  \begin{flushleft}
  \bibliography{literatura} % viz. literatura.bib
  \end{flushleft}
  \appendix
  
  \chapter{Obsah CD}

\begin{itemize}
  \item {\tt projekt.pdf} -- Technická zpráva ve formátu PDF
  \item {\tt projekt} -- Zdrojové soubory technické zprávy ve formátu \LaTeX

  \item {\tt wildfly-8.0.0.Final} -- Instalace WildFly s nainstalovanými implementovanými rošířeními, použitelná pro předvádění bakalářské práce

  \item {\tt jsm-policy-subsystem} -- Subsystém WildFly implementovaný jako hlavní součást této práce
  \item {\tt jsm-policy-console-hal} -- Rozšíření webové konzole WildFly umožňující správu výše uvedeného rozšíření (popsané v kapitole \ref{navrhGUI})
  \item {\tt jsm-policy-test} -- Repozitář integračních testů popsaných v kapitole \ref{testovani}

  \item {\tt jboss-modules} -- Záplata popsaná v kapitole \ref{upravaZavadeceWildFly} v odpovídajícím Git repozitáři
\end{itemize}


\chapter{Postup instalace}

Uvedený postup předpokládá operační systém Linux s nainstalovaným běhovým prostředím Javy. Testován byl na distribuci Ubuntu 13.10 s běhovým prostředím Javy OpenJDK 7.

\begin{enumerate}
  \item Připravte standardní instalaci aplikačního serveru WildFly. Doporučenou verzí je 8.0.0.Final. Postup by měl fungovat i pro novější verze.
  
  \item Proveďte instalaci záplaty {\tt jboss-modules}: (Její účel je popsán v kapitole \ref{zmenaZaBehu})
  \begin{enumerate}
    \item Nakopírujte do zapisovatelného adresáře repozitář {\tt jboss-modules}. Získat jej můžete ze stejně pojmenovaného adresáře na přiloženém CD nebo z GitHub:
      \newline\url{https://github.com/honza889/jboss-modules}
    \item Proveďte kompilaci za pomoci nástroje Maven -- v adresáři {\tt jboss-modules} spusťte: {\tt mvn install}
    \item Výsledným archivem ({\tt target/jboss-modules-*-SNAPSHOT.jar}) nahraďte archiv {\tt jboss-modules.jar} v adresáři WildFly.
    \item Na začátek spouštěcího skriptu, kterým budete WildFly spouštět, (např. {\tt bin/dom ain.sh}) přidejte:
      \begin{lstlisting}
JAVA_OPTS="$JAVA_OPTS -Djboss.modules.policy-refreshable=true"
      \end{lstlisting}
    \item Po dalším startu by již WildFly měl používat dynamická oprávnění. (viz kapitola \ref{staticPerm})
  \end{enumerate}
  
  \item Proveďte instalaci rozšíření WildFly {\tt jsm-policy-subsystem}:
  \begin{enumerate}
    \item Nakopírujte do zapisovatelného adresáře repozitář {\tt jsm-policy-subsystem}. Získat jej můžete opět z přiloženého CD nebo z GitHub:
      \newline\url{https://github.com/honza889/jsm-policy-subsystem}
    \item Proveďte kompilaci za pomoci nástroje Maven: {\tt mvn install}
    \item Výsledný adresář {\tt target/module/org} nahrajte do adresáře {\tt modules/system/la\linebreak yers/base/} v adresáři WildFly.
    \item Spusťe konzolu WildFly, {\tt bin/jboss-cli.sh} a následujícím příkazem nainstalujte a přidejte subsystém. Pro {\it standalone} režim:
      \begin{lstlisting}
/extension=org.picketbox.jsmpolicy.subsystem:add
/subsystem=jsmpolicy:add
      \end{lstlisting}
      Pro profil {\tt full} v režimu {\tt domain}:
      \begin{lstlisting}
/extension=org.picketbox.jsmpolicy.subsystem:add
/profile=full/subsystem=jsmpolicy:add
      \end{lstlisting}
    \item Jestliže se oba příkazy zdaří, subsystém {\tt jsmpolicy} byl úspěšně nainstalován a přidán do profilu. (V příkladu pro {\tt domain} režim do profilu {\tt full}.)
  \end{enumerate}
  
  \item Proveďte instalaci rozšíření webové konzoly WildFly {\tt jsm-policy-console-hal}:
  \begin{enumerate}
    \item Nakopírujte do zapisovatelného adresáře repozitář {\tt jsm-policy-console-hal}. Získat jej můžete z přiloženého CD nebo z GitHub:
      \newline\url{https://github.com/honza889/jsm-policy-console-hal}
    \item Proveďte kompilaci za pomoci nástroje Maven: {\tt mvn install}
    \item Výsledný archiv ({\tt build/app/target/jboss-as-console-*-resources.jar}) nahrajte do adresáře {\tt modules/system/layers/base/org/jboss/as/console/main} v adresáři WildFly.
    \item Upravte soubor {\tt module.xml}, aby atribut {\tt path} elementu {\tt resource-root} odpovídal názvu souboru zkopírovaném v minulém kroku.
    \item Proveďte restart aplikačního serveru WildFly a následně znovunačtěte webovou stránku s webovou konzolou WildFly. (Např. stisknutím klávesy F5 nebo Ctrl+F5 ve webovém prohlížeči.)
    \item Jestliže se ve webové administrační konzole v nabídce subsystémů {\it (Subsystems)} ukáže nová položka {\bf JSM Policy}, bylo toto rozšíření úspěšně nainstalováno.
  \end{enumerate}
  
\end{enumerate}


\section{Možné příčiny problémů}
\begin{enumerate}
    \item V nabídce subsystémů ve webové konzoli se nezobrazuje subsystém JSM Policy.
    \begin{enumerate}
        \item Ve webové konzole je vybraný profil, v kterém není nainstalován subsystém {\tt jsmpolicy} (dle bodu 3 postupu). (V případě {\it standalone} režimu: Není nainstalován subsystém {\tt jsmpolicy}.)
        \item Zkompilované rozšíření webové konzoly nebylo (dle bodu 4c postupu) nakopírováno do správného adresáře.
        \item V tomto adresáři nebyl upraven soubor {\tt module.xml} tak, aby odkazoval na soubor nakopírovaného rozšíření. (dle bodu 4d)
        \item Webová stránka s webovou konzolí WildFly nebyla znovunačtena nebo server WildFly restartován (dle bodu 4e postupu).
    \end{enumerate}
\end{enumerate}


\section{Spuštění integračních testů}

Jestliže jste úspěšně dokončili instalaci rozšíření aplikačního, můžete provést instalaci a spuštění integračních testů popsaných v kapitole \ref{testovani}.

\begin{enumerate}
  \item Nakopírujte do zapisovatelného adresáře repozitář {\tt jsm-policy-test}. Získat jej můžete z přiloženého CD nebo z GitHub:
    \newline\url{https://github.com/honza889/jsm-policy-test}
  \item Ujistěte se že testovací servery testovací domény WildFly s nainstalovanými rozšířeními jsou spuštěné. Integrační test využívá dvou testovacích serverů.
  \item Upravte v konfiguračním souboru nástroje Maven {\tt agent/pom.xml} údaje nezbytné pro přihlášení k testovací doméně.
  \item Upravte v souboru {\tt manager/src/test/java/org/picketbox/jsmpolicy/test/Con stants.java} informace o testovacích serverech a testovací doméně.
  \item Spusťte skript {\tt ./test.sh}
  \item Výsledky testů budou uloženy ve formě HTML dokumentu do souboru {\tt manager/targe t/site/surefire-report.html}, který by měl být po skončení testů automaticky zobrazen.
\end{enumerate}

%\chapter{Manual}
%\chapter{Konfigrační soubor}
%\chapter{RelaxNG Schéma konfiguračního soboru}
%\chapter{Plakat}

 % viz. prilohy.tex
\end{document}
